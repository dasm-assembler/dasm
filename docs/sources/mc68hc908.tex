\chapter{MC68HC908 Processor}
\label{processor:MC68HC908}
\index{Processor!68908}
\index{Processor!MC68HC908}

%% Variants of the \mono{MC68HC908} CPU are widely used in DIN-rail mountable Micro-SPS from french manufacturer Crouzet. 
%% Variant mr32  used in  Crouzet XT20 Millenium
%% Variant jb8   used in  USB08 Welcome Kit (Evaluation Board)
The \mono{MC68HC908} CPU is manufactured by Motorola/Freescale/NXP. There is a wide spectrum of CPU variants available.
\dasm has include-file support for the variants {\it{jk1}}, {\it{jk3}}, {\it{jb8}} and {\it{mr32}} in its machines/68hc908 subdirectory.
Some CPU variants can be clocked by a PLL and support advanced PWM generation for motor applications where others feature a USB 1.0 interface 
or special keyboard GPIO. Various numbers of 8- and 16bit timers for input capture / output compare operations are available as well as 10bit 
ADC converters, SPI (sometimes also UART) communication interface and brown-out reset. Onchip flash memory ranges from 2 to 32kByte, RAM ranges 
from 128 to 768 bytes and clocking speeds ranges from 1 to 8MHz.

\section{Endianness}

The \mono{MC68HC908} is a big-endian machine. Byte ordering in words is high, then low.

\section{Adressing modes}

Beside the "standard" addressing modes (listed in table \ref{standard_address_modes_68hc908}) there are 6 special addressing modes that 
apply to 6 distinct opcodes (table \ref{special_address_modes_68hc908}). These addressing modes include inherent given source- \emph{and} destination- 
address mode and optionally an increment operation for the index register. The modes dd, dix+, imd and ix+d apply to move-data mnemonics.
The modes ix+ an ix8+ apply to compare-and-branch-if-equal mnemonic.


%%  The \setlongtables command keeps column widths the same across
%%  pages. Simply comment out next line for varying column widths.
%\setlongtables


\def\standardAddressingModesCaption68hc908{Abbreviations for standard addressing modes}
\def\specialAddressingModesCaption68hc908{Abbreviations for special addressing modes}



\providecommand{\samTablePB}[1]%
{\let\samTableTemp=\\#1\let\\=\samTableTemp\hspace{0pt}}
 \ifundefined{samTableWidthDefined}
		\newlength{\samTableWidth}
		\newlength{\samTableWidthComplete}
		\global\def\samTableWidthDefined{}
 \fi

\def\samTableColAWidth{28pt}
\def\samTableColBWidth{135pt}

\setlength\samTableWidth{%
	 \samTableColAWidth+%
	 \samTableColBWidth+%
0pt}

\def\samTableNumCols{2}

\setlength\samTableWidthComplete{\samTableWidth+%
         \tabcolsep*\samTableNumCols*2+\arrayrulewidth*\samTableNumCols}
         
\ifthenelse{\lengthtest{\samTableWidthComplete > \linewidth}}%
         {\def\samTableScale{\ratio{\linewidth-%
                        \tabcolsep*\samTableNumCols*2-%
                        \arrayrulewidth*\samTableNumCols}%
{\samTableWidth}}}%
{\def\samTableScale{1}}

\ifthenelse{\isundefined{\samTableColB}}{\newlength{\samTableColB}}{}\settowidth{\samTableColB}{\begin{tabular}{@{}p{\samTableColAWidth*\samTableScale}@{}}x\end{tabular}}
\ifthenelse{\isundefined{\samTableColC}}{\newlength{\samTableColC}}{}\settowidth{\samTableColC}{\begin{tabular}{@{}p{\samTableColBWidth*\samTableScale}@{}}x\end{tabular}}

\def\standardAddressModesTable{
\begin{longtable}{|p{\samTableColAWidth}|p{\samTableColBWidth}|}
\hhline{|-|-|}
	 \samTablePB{\centering}\gnumbox{}\ninerm{inh}%
	 &\samTablePB{\raggedright}\gnumbox{}\ninerm{inherent}%
\\
\hhline{|-|~|}
	 \samTablePB{\centering}\gnumboxLG{}\ninerm{imm8}%
	 &\samTablePB{\raggedright}\gnumbox{}\ninerm{immediate byte}%
\\
\hhline{|-|~|}
	 \samTablePB{\centering}\gnumboxGN{}\ninerm{imm16}%
	&\samTablePB{\raggedright}\gnumbox{}\ninerm{immediate word}%
\\	
\hhline{|-|~|}
	 \samTablePB{\centering}\gnumboxLB{}\ninerm{rel}%
	&\samTablePB{\raggedright}\gnumbox{}\ninerm{relative}%
\\
\hhline{|-|~|}
	 \samTablePB{\centering}\gnumboxLO{}\ninerm{dir}%
	&\samTablePB{\raggedright}\gnumbox{}\ninerm{direct (zero page/8bit address)}%
\\
\hhline{|-|~|}
	 \samTablePB{\centering}\gnumbox{}\ninerm{ext}%
	&\samTablePB{\raggedright}\gnumbox{}\ninerm{16bit address (absolute)}%
\\
\hhline{|-|~|}
	 \samTablePB{\centering}\gnumbox{}\ninerm{x16}%
	&\samTablePB{\raggedright}\gnumbox{}\ninerm{X-indexed + word-offset}%
\\
\hhline{|-|~|}
	 \samTablePB{\centering}\gnumbox{}\ninerm{sp16}%
	&\samTablePB{\raggedright}\gnumbox{}\ninerm{SP-indexed + word-offset}%
\\
\hhline{|-|~|}
	 \samTablePB{\centering}\gnumboxLP{}\ninerm{x8}%
	&\samTablePB{\raggedright}\gnumbox{}\ninerm{X-indexed + byte-offset}%
\\
\hhline{|-|~|}
	 \samTablePB{\centering}\gnumboxSB{}\ninerm{sp8}%
	&\samTablePB{\raggedright}\gnumbox{}\ninerm{SP-indexed + byte-offset}%
\\
\hhline{|-|~|}
	 \samTablePB{\centering}\gnumboxLY{}\ninerm{x}%
	&\samTablePB{\raggedright}\gnumbox{}\ninerm{X-indexed}%
\\
\hhline{|-|~|}
	 \samTablePB{\centering}\gnumbox{}\ninerm{-}%
	&\samTablePB{\raggedright}\gnumbox{}\ninerm{illegal opcode}%
\\
\hhline{|-|~|}
	 \samTablePB{\centering}\gnumboxb{}\ninerm{*}%
	&\samTablePB{\raggedright}\gnumbox{}\ninerm{prefix for SP-indexed address mode}%
\\
\hhline{--}
\caption[\standardAddressingModesCaption68hc908]{\label{standard_address_modes_68hc908}\standardAddressingModesCaption68hc908}%
\end{longtable}
}





\providecommand{\smTablePB}[1]%
{\let\smTableTemp=\\#1\let\\=\smTableTemp\hspace{0pt}}
 \ifundefined{smTableWidthDefined}
        \newlength{\smTableWidth}
        \newlength{\smTableWidthComplete}
        \global\def\smTableWidthDefined{}
 \fi

%% The following setting protects this code from babel shorthands.  %%
%% \ifthenelse{\isundefined{\languageshorthands}}{}{\languageshorthands{english}}

%% to adjust positions in multirow situations                       %%
%%\setlength{\bigstrutjot}{\jot}
%%\setlength{\extrarowheight}{\doublerulesep}

%%  The \setlongtables command keeps column widths the same across  %%
%%  pages. Simply comment out next line for varying column widths.  %%
%%\setlongtables

\def\smTableColAWidth{23pt}
\def\smTableColBWidth{173pt}

\setlength\smTableWidth{%
	 \smTableColAWidth+%
	 \smTableColBWidth+%
0pt}


\def\smTableNumCols{2}
\setlength\smTableWidthComplete{\smTableWidth+%
         \tabcolsep*\smTableNumCols*2+\arrayrulewidth*\smTableNumCols}
\ifthenelse{\lengthtest{\smTableWidthComplete > \linewidth}}%
         {\def\smTableScale{\ratio{\linewidth-%
                        \tabcolsep*\smTableNumCols*2-%
                        \arrayrulewidth*\smTableNumCols}%
{\smTableWidth}}}%
{\def\smTableScale{1}}

\ifthenelse{\isundefined{\smTableColB}}{\newlength{\smTableColB}}{}\settowidth{\smTableColB}{\begin{tabular}{@{}p{\smTableColAWidth*\smTableScale}@{}}x\end{tabular}}
\ifthenelse{\isundefined{\smTableColC}}{\newlength{\smTableColC}}{}\settowidth{\smTableColC}{\begin{tabular}{@{}p{\smTableColBWidth*\smTableScale}@{}}x\end{tabular}}

\def\specialAddressModesTable{
\begin{longtable}{|p{\smTableColAWidth}|p{\smTableColBWidth}|}
\hhline{|-|-|}
	 \smTablePB{\centering}\gnumboxY{}\ninerm{dd}
	&\smTablePB{\raggedright}\gnumbox{}\ninerm{direct to direct}
\\
	 \smTablePB{\centering}\gnumboxO{}\ninerm{dix+}
	&\smTablePB{\raggedright}\gnumbox{}\ninerm{direct to X-indexed and post-increment X}
\\
	 \smTablePB{\centering}\gnumboxP{}\ninermb{imd}
	&\smTablePB{\raggedright}\gnumbox{}\ninerm{immediate to direct}
\\
	 \smTablePB{\centering}\gnumboxDG{}\ninermb{ix+d}
	&\smTablePB{\raggedright}\gnumbox{}\ninerm{X-indexed to direct and post-increment X}
\\
\hhline{|-|-|}
	 \smTablePB{\centering}\gnumboxLR{}\ninermb{ix+}
	&\smTablePB{\raggedright}\gnumbox{}\ninerm{X-indexed and post-increment X}
\\
	 \smTablePB{\centering}\gnumboxG{}\ninerm{ix8+}
	&\smTablePB{\raggedright}\gnumbox{}\ninerm{X-indexed + byte-offset and post-increment X}
\\	
\hhline{--}
\caption[\specialAddressingModesCaption68hc908]{\label{special_address_modes_68hc908}\specialAddressingModesCaption68hc908}%
\end{longtable}
}



 
\begin{table}[hbp]
\parbox[t]{.48\linewidth}{
\standardAddressModesTable
}
\hfill
\parbox[t]{.48\linewidth}{
\specialAddressModesTable
}
\end{table}

\newpage
\section{Opcode Summary}

Although dasm mnemonics are case-insensitive they are written in camel-case in table \ref{OpcodeSummary68hc908}.
This is to indicate the inherent given register(s) for the first\footnote{and second operand if given} operand:

\begin{itemize}
\item 8bit-register A
\item 8bit-register X
\item 8bit-register H
\item 16bit-register HX (combined of 8bit-register H and X)
\item 8bit-register P (stack-pointer low byte)
\item 16bit-register S (stack-pointer)
\end{itemize}

There are two opcodes that need operand arguments which don't fit into any scheme
implemented in \dasm. However they can be generated by using 
forced\footnote{see \ref{pseudoop:force}} addressing mode. 
Extension {\it{.ix}} (implied, X-indexed) has to be used in this case.

\begin{table}
\begin{tabular}{ccp{0.5\linewidth}}
Opcode & Mnemonic & Description
\\
\hhline{---}
{\mono{\$71}} & {\mono{cbeq{\it{.ix}}}} & \makebox[2pt]{}{compare register A with value indexed by X, branch relative if not equal and post-increment\footnote{note that also opcode {\mono{\$61}} does post-increment X} X}
\\
{\mono{\$7B}} & {\mono{dbnz{\it{.ix}}}} & \makebox[2pt]{}{decrement value indexed by X and branch relative if not zero.}
\end{tabular}
\caption{}
\end{table}

\newgeometry{top=30mm, bottom=30mm, left=25mm, right=15mm}%
\begin{landscape}%
{%
	
%%%%%%%%%%%%%%%%%%%%%%%%%%%%%%%%%%%%%%%%%%%%%%%%%%%%%%%%%%%%%%%%%%%%%%
%%                                                                  %%
%%  The rest is the gnumeric table, except for the closing          %%
%%  statement. Changes below will alter the table's appearance.     %%
%%                                                                  %%
%%%%%%%%%%%%%%%%%%%%%%%%%%%%%%%%%%%%%%%%%%%%%%%%%%%%%%%%%%%%%%%%%%%%%%

%% \providecommand{\gnumericmathit}[1]{#1} 

%%  Uncomment the next line if you would like your numbers to be in %%
%%  italics if they are italizised in the gnumeric table.           %%
%% \renewcommand{\gnumericmathit}[1]{\mathit{#1}}

\providecommand{\tableOmPB}[1]%
{\let\tableOmTemp=\\#1\let\\=\tableOmTemp\hspace{0pt}}
 \ifundefined{tableOmWidthDefined}
		\newlength{\tableOmWidth}
        \newlength{\tableOmWidthComplete}
		\global\def\tableOmWidthDefined{}
 \fi

%% The following setting protects this code from babel shorthands.  %%

%%  The default table format retains the relative column widths of  %%
%%  gnumeric. They can easily be changed to c, r or l. In that case %%
%%  you may want to comment out the next line and uncomment the one %%
%%  thereafter                                                      %%

%% to adjust positions in multirow situations                       %%
\setlength{\bigstrutjot}{\jot}
\setlength{\extrarowheight}{\doublerulesep}

%%  The \setlongtables command keeps column widths the same across  %%
%%  pages. Simply comment out next line for varying column widths.  %%
%%\setlongtables

\setlength\tableOmWidth{%
	60pt+%
	50pt+%
	50pt+%
	60pt+%
	50pt+%
	50pt+%
	50pt+%
	50pt+%
	50pt+%
	50pt+%
	50pt+%
	50pt+%
	50pt+%
	50pt+%
	50pt+%
	50pt+%
	50pt+%
	50pt+%
	50pt+%
	50pt+%
0pt}

\def\tableOmNumCols{20}

\setlength\tableOmWidthComplete{\tableOmWidth+%
         \tabcolsep*\tableOmNumCols*2+\arrayrulewidth*\tableOmNumCols}
         
\ifthenelse{\lengthtest{\tableOmWidthComplete > \linewidth}}%
         {\def\tableOmScale{\ratio{\linewidth-%
                        \tabcolsep*\tableOmNumCols*2-%
                        \arrayrulewidth*\tableOmNumCols}%
						{\tableOmWidth}}%
		 }%
		 {
		 	\def\tableOmScale{1}%
		 }

\ifthenelse{\isundefined{\tableOmColB}}{\newlength{\tableOmColB}}{}\settowidth{\tableOmColB}{\begin{tabular}{@{}p{60pt*\tableOmScale}@{}}x\end{tabular}}
\ifthenelse{\isundefined{\tableOmColC}}{\newlength{\tableOmColC}}{}\settowidth{\tableOmColC}{\begin{tabular}{@{}p{50pt*\tableOmScale}@{}}x\end{tabular}}
\ifthenelse{\isundefined{\tableOmColD}}{\newlength{\tableOmColD}}{}\settowidth{\tableOmColD}{\begin{tabular}{@{}p{50pt*\tableOmScale}@{}}x\end{tabular}}
\ifthenelse{\isundefined{\tableOmColE}}{\newlength{\tableOmColE}}{}\settowidth{\tableOmColE}{\begin{tabular}{@{}p{60pt*\tableOmScale}@{}}x\end{tabular}}
\ifthenelse{\isundefined{\tableOmColF}}{\newlength{\tableOmColF}}{}\settowidth{\tableOmColF}{\begin{tabular}{@{}p{50pt*\tableOmScale}@{}}x\end{tabular}}
\ifthenelse{\isundefined{\tableOmColG}}{\newlength{\tableOmColG}}{}\settowidth{\tableOmColG}{\begin{tabular}{@{}p{50pt*\tableOmScale}@{}}x\end{tabular}}
\ifthenelse{\isundefined{\tableOmColH}}{\newlength{\tableOmColH}}{}\settowidth{\tableOmColH}{\begin{tabular}{@{}p{50pt*\tableOmScale}@{}}x\end{tabular}}
\ifthenelse{\isundefined{\tableOmColI}}{\newlength{\tableOmColI}}{}\settowidth{\tableOmColI}{\begin{tabular}{@{}p{50pt*\tableOmScale}@{}}x\end{tabular}}
\ifthenelse{\isundefined{\tableOmColJ}}{\newlength{\tableOmColJ}}{}\settowidth{\tableOmColJ}{\begin{tabular}{@{}p{50pt*\tableOmScale}@{}}x\end{tabular}}
\ifthenelse{\isundefined{\tableOmColK}}{\newlength{\tableOmColK}}{}\settowidth{\tableOmColK}{\begin{tabular}{@{}p{50pt*\tableOmScale}@{}}x\end{tabular}}
\ifthenelse{\isundefined{\tableOmColL}}{\newlength{\tableOmColL}}{}\settowidth{\tableOmColL}{\begin{tabular}{@{}p{50pt*\tableOmScale}@{}}x\end{tabular}}
\ifthenelse{\isundefined{\tableOmColM}}{\newlength{\tableOmColM}}{}\settowidth{\tableOmColM}{\begin{tabular}{@{}p{50pt*\tableOmScale}@{}}x\end{tabular}}
\ifthenelse{\isundefined{\tableOmColN}}{\newlength{\tableOmColN}}{}\settowidth{\tableOmColN}{\begin{tabular}{@{}p{50pt*\tableOmScale}@{}}x\end{tabular}}
\ifthenelse{\isundefined{\tableOmColO}}{\newlength{\tableOmColO}}{}\settowidth{\tableOmColO}{\begin{tabular}{@{}p{50pt*\tableOmScale}@{}}x\end{tabular}}
\ifthenelse{\isundefined{\tableOmColP}}{\newlength{\tableOmColP}}{}\settowidth{\tableOmColP}{\begin{tabular}{@{}p{50pt*\tableOmScale}@{}}x\end{tabular}}
\ifthenelse{\isundefined{\tableOmColQ}}{\newlength{\tableOmColQ}}{}\settowidth{\tableOmColQ}{\begin{tabular}{@{}p{50pt*\tableOmScale}@{}}x\end{tabular}}
\ifthenelse{\isundefined{\tableOmColR}}{\newlength{\tableOmColR}}{}\settowidth{\tableOmColR}{\begin{tabular}{@{}p{50pt*\tableOmScale}@{}}x\end{tabular}}
\ifthenelse{\isundefined{\tableOmColS}}{\newlength{\tableOmColS}}{}\settowidth{\tableOmColS}{\begin{tabular}{@{}p{50pt*\tableOmScale}@{}}x\end{tabular}}
\ifthenelse{\isundefined{\tableOmColT}}{\newlength{\tableOmColT}}{}\settowidth{\tableOmColT}{\begin{tabular}{@{}p{50pt*\tableOmScale}@{}}x\end{tabular}}
\ifthenelse{\isundefined{\tableOmColU}}{\newlength{\tableOmColU}}{}\settowidth{\tableOmColU}{\begin{tabular}{@{}p{50pt*\tableOmScale}@{}}x\end{tabular}}

%%\def\opcodeSummaryTable{
\begin{longtable}[c]{%
	b{\tableOmColB}%
	b{\tableOmColC}%
	b{\tableOmColD}%
	b{\tableOmColE}%
	b{\tableOmColF}%
	b{\tableOmColG}%
	b{\tableOmColH}%
	b{\tableOmColI}%
	b{\tableOmColJ}%
	b{\tableOmColK}%
	b{\tableOmColL}%
	b{\tableOmColM}%
	b{\tableOmColN}%
	b{\tableOmColO}%
	b{\tableOmColP}%
	b{\tableOmColQ}%
	b{\tableOmColR}%
	b{\tableOmColS}%
	b{\tableOmColT}%
	b{\tableOmColU}%
	}%
\hhline{~|--|-|------|--|--------}
	 \multicolumn{1}{p{\tableOmColB}|}%
	{}
	&\multicolumn{2}{p{	\tableOmColC+%
	\tableOmColD+%
	\tabcolsep*2*1}|}%
	{\tableOmPB{\centering}\eightrm{Bit Manipulation}}
	&\multicolumn{1}{p{\tableOmColE}|}%
	{\tableOmPB{\centering}\gnumbox{\eightrm{Branch}}}
	&
	&\tableOmPB{\raggedright}\gnumbox{\ninerm{\makebox[5pt]{}Read-Modify-Write}}
	&
	&
	&
	&\multicolumn{1}{p{\tableOmColK}|}%
	{}
	&\tableOmPB{\centering}\gnumbox{\ninerm{\makebox[5pt]{}Control}}
	&\multicolumn{1}{p{\tableOmColM}|}%
	{}
	&\tableOmPB{\raggedright}\gnumbox[l]{\ninerm{Register-Memory}}
	&
	&
	&
	&
	&
	&
	&\multicolumn{1}{p{\tableOmColU}|}%
	{}
\\
\hhline{~|--|-|------|--|--------|}
	&\tableOmPB{\centering}\gnumboxLO{\ninerm{dir}}
	&\tableOmPB{\centering}\gnumboxLO{\ninerm{dir}}
	&\tableOmPB{\centering}\gnumboxLB{\ninerm{rel}}
	&\tableOmPB{\centering}\gnumboxLO{\ninerm{dir}}
	&\tableOmPB{\centering}\gnumbox{\ninerm{inh}}
	&\tableOmPB{\centering}\gnumbox{\ninerm{inh}}
	&\tableOmPB{\centering}\gnumboxLP{\ninerm{x8}}
	&\tableOmPB{\centering}\gnumboxSB{\ninerm{sp8}}
	&\tableOmPB{\centering}\gnumboxLY{\ninerm{x}}
	&\tableOmPB{\centering}\gnumbox{\ninerm{inh}}
	&\tableOmPB{\centering}\gnumbox{\ninerm{inh}}
	&\tableOmPB{\centering}\gnumboxLG{\ninerm{imm8}}
	&\tableOmPB{\centering}\gnumboxLO{\ninerm{dir}}
	&\tableOmPB{\centering}\gnumbox{\ninerm{ext}}
	&\tableOmPB{\centering}\gnumbox{\ninerm{x16}}
	&\tableOmPB{\centering}\gnumbox{\ninerm{sp16}}
	&\tableOmPB{\centering}\gnumboxLP{\ninerm{x8}}
	&\tableOmPB{\centering}\gnumboxSB{\ninerm{sp8}}
	&\tableOmPB{\centering}\gnumboxLY{\ninerm{x}}
\\
\hhline{|-|-------------------|}
	 \multicolumn{1}{|p{\tableOmColB}}%
%%	 {\tableOmPB{\centering}\gnumbox{\diagbox[trim=lr,width=13mm,height=10mm]{\adjustbox{valign=b}{\sevenrm{LSB}}}{\adjustbox{valign=t}{\sevenrm{MSB}}}}}%
	 {\tableOmPB{\centering}\gnumbox{\diagbox[trim=lr,width=12.75mm,height=7mm]{\adjustbox{valign=b}{\sevenrm{LSB}}}{\adjustbox{valign=t}{\sevenrm{MSB}}}}}%
	&\tableOmPB{\centering}\gnumbox{\ninerm{0{\color{ghost}x}}}%
	&\tableOmPB{\centering}\gnumbox{\ninerm{1{\color{ghost}x}}}%
	&\tableOmPB{\centering}\gnumbox{\ninerm{2{\color{ghost}x}}}%
	&\tableOmPB{\centering}\gnumbox{\ninerm{3{\color{ghost}x}}}%
	&\tableOmPB{\centering}\gnumbox{\ninerm{4{\color{ghost}x}}}%
	&\tableOmPB{\centering}\gnumbox{\ninerm{5{\color{ghost}x}}}%
	&\tableOmPB{\centering}\gnumbox{\ninerm{6{\color{ghost}x}}}%
	&\tableOmPB{\centering}\gnumbox{\ninerm{9e6{\color{ghost}x}}}%
	&\tableOmPB{\centering}\gnumbox{\ninerm{7{\color{ghost}x}}}%
	&\tableOmPB{\centering}\gnumbox{\ninerm{8{\color{ghost}x}}}%
	&\tableOmPB{\centering}\gnumbox{\ninerm{9{\color{ghost}x}}}%
	&\tableOmPB{\centering}\gnumbox{\ninerm{A{\color{ghost}x}}}%
	&\tableOmPB{\centering}\gnumbox{\ninerm{B{\color{ghost}x}}}%
	&\tableOmPB{\centering}\gnumbox{\ninerm{C{\color{ghost}x}}}%
	&\tableOmPB{\centering}\gnumbox{\ninerm{D{\color{ghost}x}}}%
	&\tableOmPB{\centering}\gnumbox{\ninerm{9eD{\color{ghost}x}}}%
	&\tableOmPB{\centering}\gnumbox{\ninerm{E{\color{ghost}x}}}%
	&\tableOmPB{\centering}\gnumbox{\ninerm{9eE{\color{ghost}x}}}%
	&\multicolumn{1}{p{\tableOmColU}|}%
	{\tableOmPB{\centering}\gnumbox{\ninerm{F{\color{ghost}x}}}}%
\\
\hhline{~|-------------------|}
	 \multicolumn{1}{|p{\tableOmColB}|}%
	{\tableOmPB{\centering}\gnumbox{{\ninerm{\color{ghost}x}0}}}
	&\tableOmPB{\centering}\gnumboxLO{\ninerm{brset 0}}
	&\tableOmPB{\centering}\gnumboxLO{\ninerm{bset 0}}
	&\tableOmPB{\centering}\gnumboxLB{\ninerm{bra}}
	&\tableOmPB{\centering}\gnumboxLO{\ninerm{neg}}
	&\tableOmPB{\centering}\gnumbox{\ninerm{negA}}
	&\tableOmPB{\centering}\gnumbox{\ninerm{negX}}
	&\tableOmPB{\centering}\gnumboxLP{\ninerm{neg}}
	&\tableOmPB{\centering}\gnumboxSB{\ninerm{neg}}
	&\tableOmPB{\centering}\gnumboxLY{\ninerm{neg}}
	&\tableOmPB{\centering}\gnumbox{\ninerm{rti}}
	&\tableOmPB{\centering}\gnumboxLB{\ninerm{bge}}
	&\tableOmPB{\centering}\gnumboxLG{\ninerm{sub}}
	&\tableOmPB{\centering}\gnumboxLO{\ninerm{sub}}
	&\tableOmPB{\centering}\gnumbox{\ninerm{sub}}
	&\tableOmPB{\centering}\gnumbox{\ninerm{sub}}
	&\tableOmPB{\centering}\gnumbox{\ninerm{sub}}
	&\tableOmPB{\centering}\gnumboxLP{\ninerm{sub}}
	&\tableOmPB{\centering}\gnumboxSB{\ninerm{sub}}
	&\tableOmPB{\centering}\gnumboxLY{\ninerm{sub}}
\\
	 \multicolumn{1}{|p{\tableOmColB}|}%
	{\tableOmPB{\centering}\gnumbox{\ninerm{{\color{ghost}x}1}}}
	&\tableOmPB{\centering}\gnumboxLO{\ninerm{brclr 0}}
	&\tableOmPB{\centering}\gnumboxLO{\ninerm{bclr 0}}
	&\tableOmPB{\centering}\gnumboxLB{\ninerm{brn}}
	&\tableOmPB{\centering}\gnumboxLO{\ninerm{cbeq}}
	&\tableOmPB{\centering}\gnumboxLG{\ninerm{cbeqA}}
	&\tableOmPB{\centering}\gnumboxLG{\ninerm{cbeqX}}
	&\tableOmPB{\centering}\gnumboxLR{\ninermb{cbeq}}
	&\tableOmPB{\centering}\gnumboxSB{\ninerm{cbeq}}
	&\tableOmPB{\centering}\gnumboxG{\ninerm{cbeq.ix}}
	&\tableOmPB{\centering}\gnumbox{\ninerm{rts}}
	&\tableOmPB{\centering}\gnumboxLB{\ninerm{blt}}
	&\tableOmPB{\centering}\gnumboxLG{\ninerm{cmp}}
	&\tableOmPB{\centering}\gnumboxLO{\ninerm{cmp}}
	&\tableOmPB{\centering}\gnumbox{\ninerm{cmp}}
	&\tableOmPB{\centering}\gnumbox{\ninerm{cmp}}
	&\tableOmPB{\centering}\gnumbox{\ninerm{cmp}}
	&\tableOmPB{\centering}\gnumboxLP{\ninerm{cmp}}
	&\tableOmPB{\centering}\gnumboxSB{\ninerm{cmp}}
	&\tableOmPB{\centering}\gnumboxLY{\ninerm{cmp}}
\\
	 \multicolumn{1}{|p{\tableOmColB}|}%
	{\tableOmPB{\centering}\gnumbox{\ninerm{{\color{ghost}x}2}}}
	&\tableOmPB{\centering}\gnumboxLO{\ninerm{brset 1}}
	&\tableOmPB{\centering}\gnumboxLO{\ninerm{bset 1}}
	&\tableOmPB{\centering}\gnumboxLB{\ninerm{bhi}}
	&\tableOmPB{\centering}\gnumbox{\ninerm{-}}
	&\tableOmPB{\centering}\gnumbox{\ninerm{mul}}
	&\tableOmPB{\centering}\gnumbox{\ninerm{div}}
	&\tableOmPB{\centering}\gnumbox{\ninerm{nsa}}
	&\tableOmPB{\centering}\gnumbox{\ninerm{-}}
	&\tableOmPB{\centering}\gnumbox{\ninerm{daa}}
	&\tableOmPB{\centering}\gnumbox{\ninerm{-}}
	&\tableOmPB{\centering}\gnumboxLB{\ninerm{bgt}}
	&\tableOmPB{\centering}\gnumboxLG{\ninerm{sbc}}
	&\tableOmPB{\centering}\gnumboxLO{\ninerm{sbc}}
	&\tableOmPB{\centering}\gnumbox{\ninerm{sbc}}
	&\tableOmPB{\centering}\gnumbox{\ninerm{sbc}}
	&\tableOmPB{\centering}\gnumbox{\ninerm{sbc}}
	&\tableOmPB{\centering}\gnumboxLP{\ninerm{sbc}}
	&\tableOmPB{\centering}\gnumboxSB{\ninerm{sbc}}
	&\tableOmPB{\centering}\gnumboxLY{\ninerm{sbc}}
\\
	 \multicolumn{1}{|p{\tableOmColB}|}%
	{\tableOmPB{\centering}\gnumbox{\ninerm{{\color{ghost}x}3}}}
	&\tableOmPB{\centering}\gnumboxLO{\ninerm{brclr 1}}
	&\tableOmPB{\centering}\gnumboxLO{\ninerm{bclr 1}}
	&\tableOmPB{\centering}\gnumboxLB{\ninerm{bls}}
	&\tableOmPB{\centering}\gnumboxLO{\ninerm{com}}
	&\tableOmPB{\centering}\gnumbox{\ninerm{comA}}
	&\tableOmPB{\centering}\gnumbox{\ninerm{comX}}
	&\tableOmPB{\centering}\gnumboxLP{\ninerm{com}}
	&\tableOmPB{\centering}\gnumboxSB{\ninerm{com}}
	&\tableOmPB{\centering}\gnumboxLY{\ninerm{com}}
	&\tableOmPB{\centering}\gnumbox{\ninerm{swi}}
	&\tableOmPB{\centering}\gnumboxLB{\ninerm{ble}}
	&\tableOmPB{\centering}\gnumboxLG{\ninerm{cpX}}
	&\tableOmPB{\centering}\gnumboxLO{\ninerm{cpX}}
	&\tableOmPB{\centering}\gnumbox{\ninerm{cpX}}
	&\tableOmPB{\centering}\gnumbox{\ninerm{cpX}}
	&\tableOmPB{\centering}\gnumbox{\ninerm{cpX}}
	&\tableOmPB{\centering}\gnumboxLP{\ninerm{cpX}}
	&\tableOmPB{\centering}\gnumboxSB{\ninerm{cpX}}
	&\tableOmPB{\centering}\gnumboxLY{\ninerm{cpX}}
\\
	 \multicolumn{1}{|p{\tableOmColB}|}%
	{\tableOmPB{\centering}\gnumbox{\ninerm{{\color{ghost}x}4}}}
	&\tableOmPB{\centering}\gnumboxLO{\ninerm{brset 2}}
	&\tableOmPB{\centering}\gnumboxLO{\ninerm{bset 2}}
	&\tableOmPB{\centering}\gnumboxLB{\ninerm{bcc}}
	&\tableOmPB{\centering}\gnumboxLO{\ninerm{lsr}}
	&\tableOmPB{\centering}\gnumbox{\ninerm{lsrA}}
	&\tableOmPB{\centering}\gnumbox{\ninerm{lsrX}}
	&\tableOmPB{\centering}\gnumboxLP{\ninerm{lsr}}
	&\tableOmPB{\centering}\gnumboxSB{\ninerm{lsr}}
	&\tableOmPB{\centering}\gnumboxLY{\ninerm{lsr}}
	&\tableOmPB{\centering}\gnumbox{\ninerm{tAP}}
	&\tableOmPB{\centering}\gnumbox{\ninerm{tHXS}}
	&\tableOmPB{\centering}\gnumboxLG{\ninerm{and}}
	&\tableOmPB{\centering}\gnumboxLO{\ninerm{and}}
	&\tableOmPB{\centering}\gnumbox{\ninerm{and}}
	&\tableOmPB{\centering}\gnumbox{\ninerm{and}}
	&\tableOmPB{\centering}\gnumbox{\ninerm{and}}
	&\tableOmPB{\centering}\gnumboxLP{\ninerm{and}}
	&\tableOmPB{\centering}\gnumboxSB{\ninerm{and}}
	&\tableOmPB{\centering}\gnumboxLY{\ninerm{and}}
\\
	 \multicolumn{1}{|p{\tableOmColB}|}%
	{\tableOmPB{\centering}\gnumbox{\ninerm{{\color{ghost}x}5}}}
	&\tableOmPB{\centering}\gnumboxLO{\ninerm{brclr 2}}
	&\tableOmPB{\centering}\gnumboxLO{\ninerm{bclr 2}}
	&\tableOmPB{\centering}\gnumboxLB{\ninerm{blo\footnote{or bcs (branch if carry set)}}}
	&\tableOmPB{\centering}\gnumboxLO{\ninerm{stHX}}
	&\tableOmPB{\centering}\gnumboxGN{\ninerm{ldHX}}
	&\tableOmPB{\centering}\gnumboxLO{\ninerm{ldHX}}
	&\tableOmPB{\centering}\gnumboxGN{\ninerm{cpHX}}
	&\tableOmPB{\centering}\gnumbox{\ninerm{-}}
	&\tableOmPB{\centering}\gnumboxLO{\ninerm{cpHX}}
	&\tableOmPB{\centering}\gnumbox{\ninerm{tPA}}
	&\tableOmPB{\centering}\gnumbox{\ninerm{tSHX}}
	&\tableOmPB{\centering}\gnumboxLG{\ninerm{bit}}
	&\tableOmPB{\centering}\gnumboxLO{\ninerm{bit}}
	&\tableOmPB{\centering}\gnumbox{\ninerm{bit}}
	&\tableOmPB{\centering}\gnumbox{\ninerm{bit}}
	&\tableOmPB{\centering}\gnumbox{\ninerm{bit}}
	&\tableOmPB{\centering}\gnumboxLP{\ninerm{bit}}
	&\tableOmPB{\centering}\gnumboxSB{\ninerm{bit}}
	&\tableOmPB{\centering}\gnumboxLY{\ninerm{bit}}
\\
	 \multicolumn{1}{|p{\tableOmColB}|}%
	{\tableOmPB{\centering}\gnumbox{\ninerm{{\color{ghost}x}6}}}
	&\tableOmPB{\centering}\gnumboxLO{\ninerm{brset 3}}
	&\tableOmPB{\centering}\gnumboxLO{\ninerm{bset 3}}
	&\tableOmPB{\centering}\gnumboxLB{\ninerm{bne}}
	&\tableOmPB{\centering}\gnumboxLO{\ninerm{ror}}
	&\tableOmPB{\centering}\gnumbox{\ninerm{rorA}}
	&\tableOmPB{\centering}\gnumbox{\ninerm{rorX}}
	&\tableOmPB{\centering}\gnumboxLP{\ninerm{ror}}
	&\tableOmPB{\centering}\gnumboxSB{\ninerm{ror}}
	&\tableOmPB{\centering}\gnumboxLY{\ninerm{ror}}
	&\tableOmPB{\centering}\gnumbox{\ninerm{pulA}}
	&\tableOmPB{\centering}\gnumbox{\ninerm{-}}
	&\tableOmPB{\centering}\gnumboxLG{\ninerm{ldA}}
	&\tableOmPB{\centering}\gnumboxLO{\ninerm{ldA}}
	&\tableOmPB{\centering}\gnumbox{\ninerm{ldA}}
	&\tableOmPB{\centering}\gnumbox{\ninerm{ldA}}
	&\tableOmPB{\centering}\gnumbox{\ninerm{ldA}}
	&\tableOmPB{\centering}\gnumboxLP{\ninerm{ldA}}
	&\tableOmPB{\centering}\gnumboxSB{\ninerm{ldA}}
	&\tableOmPB{\centering}\gnumboxLY{\ninerm{ldA}}
\\
	 \multicolumn{1}{|p{\tableOmColB}|}%
	{\tableOmPB{\centering}\gnumbox{\ninerm{{\color{ghost}x}7}}}
	&\tableOmPB{\centering}\gnumboxLO{\ninerm{brclr 3}}
	&\tableOmPB{\centering}\gnumboxLO{\ninerm{bclr 3}}
	&\tableOmPB{\centering}\gnumboxLB{\ninerm{beq}}
	&\tableOmPB{\centering}\gnumboxLO{\ninerm{asr}}
	&\tableOmPB{\centering}\gnumbox{\ninerm{asrA}}
	&\tableOmPB{\centering}\gnumbox{\ninerm{asrX}}
	&\tableOmPB{\centering}\gnumboxLP{\ninerm{asr}}
	&\tableOmPB{\centering}\gnumboxSB{\ninerm{asr}}
	&\tableOmPB{\centering}\gnumboxLY{\ninerm{asr}}
	&\tableOmPB{\centering}\gnumbox{\ninerm{pshA}}
	&\tableOmPB{\centering}\gnumbox{\ninerm{tAX}}
	&\tableOmPB{\centering}\gnumboxLG{\ninerm{aiS}}
	&\tableOmPB{\centering}\gnumboxLO{\ninerm{stA}}
	&\tableOmPB{\centering}\gnumbox{\ninerm{stA}}
	&\tableOmPB{\centering}\gnumbox{\ninerm{stA}}
	&\tableOmPB{\centering}\gnumbox{\ninerm{stA}}
	&\tableOmPB{\centering}\gnumboxLP{\ninerm{stA}}
	&\tableOmPB{\centering}\gnumboxSB{\ninerm{stA}}
	&\tableOmPB{\centering}\gnumboxLY{\ninerm{stA}}
\\
	 \multicolumn{1}{|p{\tableOmColB}|}%
	{\tableOmPB{\centering}\gnumbox{\ninerm{{\color{ghost}x}8}}}
	&\tableOmPB{\centering}\gnumboxLO{\ninerm{brset 4}}
	&\tableOmPB{\centering}\gnumboxLO{\ninerm{bset 4}}
	&\tableOmPB{\centering}\gnumboxLB{\ninerm{bhcc}}
	&\tableOmPB{\centering}\gnumboxLO{\ninerm{lsl}}
	&\tableOmPB{\centering}\gnumbox{\ninerm{lslA}}
	&\tableOmPB{\centering}\gnumbox{\ninerm{lslX}}
	&\tableOmPB{\centering}\gnumboxLP{\ninerm{lsl}}
	&\tableOmPB{\centering}\gnumboxSB{\ninerm{lsl}}
	&\tableOmPB{\centering}\gnumboxLY{\ninerm{lsl}}
	&\tableOmPB{\centering}\gnumbox{\ninerm{pulX}}
	&\tableOmPB{\centering}\gnumbox{\ninerm{clc}}
	&\tableOmPB{\centering}\gnumboxLG{\ninerm{eor}}
	&\tableOmPB{\centering}\gnumboxLO{\ninerm{eor}}
	&\tableOmPB{\centering}\gnumbox{\ninerm{eor}}
	&\tableOmPB{\centering}\gnumbox{\ninerm{eor}}
	&\tableOmPB{\centering}\gnumbox{\ninerm{eor}}
	&\tableOmPB{\centering}\gnumboxLP{\ninerm{eor}}
	&\tableOmPB{\centering}\gnumboxSB{\ninerm{eor}}
	&\tableOmPB{\centering}\gnumboxLY{\ninerm{eor}}
\\
	 \multicolumn{1}{|p{\tableOmColB}|}%
	{\tableOmPB{\centering}\gnumbox{\ninerm{{\color{ghost}x}9}}}
	&\tableOmPB{\centering}\gnumboxLO{\ninerm{brclr 4}}
	&\tableOmPB{\centering}\gnumboxLO{\ninerm{bclr 4}}
	&\tableOmPB{\centering}\gnumboxLB{\ninerm{bhcs}}
	&\tableOmPB{\centering}\gnumboxLO{\ninerm{rol}}
	&\tableOmPB{\centering}\gnumbox{\ninerm{rolA}}
	&\tableOmPB{\centering}\gnumbox{\ninerm{rolX}}
	&\tableOmPB{\centering}\gnumboxLP{\ninerm{rol}}
	&\tableOmPB{\centering}\gnumboxSB{\ninerm{rol}}
	&\tableOmPB{\centering}\gnumboxLY{\ninerm{rol}}
	&\tableOmPB{\centering}\gnumbox{\ninerm{pshX}}
	&\tableOmPB{\centering}\gnumbox{\ninerm{sec}}
	&\tableOmPB{\centering}\gnumboxLG{\ninerm{adc}}
	&\tableOmPB{\centering}\gnumboxLO{\ninerm{adc}}
	&\tableOmPB{\centering}\gnumbox{\ninerm{adc}}
	&\tableOmPB{\centering}\gnumbox{\ninerm{adc}}
	&\tableOmPB{\centering}\gnumbox{\ninerm{adc}}
	&\tableOmPB{\centering}\gnumboxLP{\ninerm{adc}}
	&\tableOmPB{\centering}\gnumboxSB{\ninerm{adc}}
	&\tableOmPB{\centering}\gnumboxLY{\ninerm{adc}}
\\
	 \multicolumn{1}{|p{\tableOmColB}|}%
	{\tableOmPB{\centering}\gnumbox{\ninerm{{\color{ghost}x}A}}}
	&\tableOmPB{\centering}\gnumboxLO{\ninerm{brset 5}}
	&\tableOmPB{\centering}\gnumboxLO{\ninerm{bset 5}}
	&\tableOmPB{\centering}\gnumboxLB{\ninerm{bpl}}
	&\tableOmPB{\centering}\gnumboxLO{\ninerm{dec}}
	&\tableOmPB{\centering}\gnumbox{\ninerm{decA}}
	&\tableOmPB{\centering}\gnumbox{\ninerm{decX}}
	&\tableOmPB{\centering}\gnumboxLP{\ninerm{dec}}
	&\tableOmPB{\centering}\gnumboxSB{\ninerm{dec}}
	&\tableOmPB{\centering}\gnumboxLY{\ninerm{dec}}
	&\tableOmPB{\centering}\gnumbox{\ninerm{pulH}}
	&\tableOmPB{\centering}\gnumbox{\ninerm{cli}}
	&\tableOmPB{\centering}\gnumboxLG{\ninerm{orA}}
	&\tableOmPB{\centering}\gnumboxLO{\ninerm{orA}}
	&\tableOmPB{\centering}\gnumbox{\ninerm{orA}}
	&\tableOmPB{\centering}\gnumbox{\ninerm{orA}}
	&\tableOmPB{\centering}\gnumbox{\ninerm{orA}}
	&\tableOmPB{\centering}\gnumboxLP{\ninerm{orA}}
	&\tableOmPB{\centering}\gnumboxSB{\ninerm{orA}}
	&\tableOmPB{\centering}\gnumboxLY{\ninerm{orA}}
\\
	 \multicolumn{1}{|p{\tableOmColB}|}%
	{\tableOmPB{\centering}\gnumbox{\ninerm{{\color{ghost}x}B}}}
	&\tableOmPB{\centering}\gnumboxLO{\ninerm{brclr 5}}
	&\tableOmPB{\centering}\gnumboxLO{\ninerm{bclr 5}}
	&\tableOmPB{\centering}\gnumboxLB{\ninerm{bmi}}
	&\tableOmPB{\centering}\gnumboxLO{\ninerm{dbnz}}
	&\tableOmPB{\centering}\gnumbox{\ninerm{dbnzA}}
	&\tableOmPB{\centering}\gnumbox{\ninerm{dbnzX}}
	&\tableOmPB{\centering}\gnumboxLP{\ninerm{dbnz}}
	&\tableOmPB{\centering}\gnumboxSB{\ninerm{dbnz}}
	&\tableOmPB{\centering}\gnumboxLY{\ninerm{dbnz.ix}}
	&\tableOmPB{\centering}\gnumbox{\ninerm{pshH}}
	&\tableOmPB{\centering}\gnumbox{\ninerm{sei}}
	&\tableOmPB{\centering}\gnumboxLG{\ninerm{add}}
	&\tableOmPB{\centering}\gnumboxLO{\ninerm{add}}
	&\tableOmPB{\centering}\gnumbox{\ninerm{add}}
	&\tableOmPB{\centering}\gnumbox{\ninerm{add}}
	&\tableOmPB{\centering}\gnumbox{\ninerm{add}}
	&\tableOmPB{\centering}\gnumboxLP{\ninerm{add}}
	&\tableOmPB{\centering}\gnumboxSB{\ninerm{add}}
	&\tableOmPB{\centering}\gnumboxLY{\ninerm{add}}
\\
	 \multicolumn{1}{|p{\tableOmColB}|}%
	{\tableOmPB{\centering}\gnumbox{\ninerm{{\color{ghost}x}C}}}
	&\tableOmPB{\centering}\gnumboxLO{\ninerm{brset 6}}
	&\tableOmPB{\centering}\gnumboxLO{\ninerm{bset 6}}
	&\tableOmPB{\centering}\gnumboxLB{\ninerm{bmc}}
	&\tableOmPB{\centering}\gnumboxLO{\ninerm{inc}}
	&\tableOmPB{\centering}\gnumbox{\ninerm{incA}}
	&\tableOmPB{\centering}\gnumbox{\ninerm{incX}}
	&\tableOmPB{\centering}\gnumboxLP{\ninerm{inc}}
	&\tableOmPB{\centering}\gnumboxSB{\ninerm{inc}}
	&\tableOmPB{\centering}\gnumboxLY{\ninerm{inc}}
	&\tableOmPB{\centering}\gnumbox{\ninerm{clrH}}
	&\tableOmPB{\centering}\gnumbox{\ninerm{rsp}}
	&\tableOmPB{\centering}\gnumbox{\ninerm{-}}
	&\tableOmPB{\centering}\gnumboxLO{\ninerm{jmp}}
	&\tableOmPB{\centering}\gnumbox{\ninerm{jmp}}
	&\tableOmPB{\centering}\gnumbox{\ninerm{jmp}}
	&\tableOmPB{\centering}\gnumbox{\ninerm{-}}
	&\tableOmPB{\centering}\gnumboxLP{\ninerm{jmp}}
	&\tableOmPB{\centering}\gnumbox{\ninerm{-}}
	&\tableOmPB{\centering}\gnumboxLY{\ninerm{jmp}}
\\
	 \multicolumn{1}{|p{\tableOmColB}|}%
	{\tableOmPB{\centering}\gnumbox{\ninerm{{\color{ghost}x}D}}}
	&\tableOmPB{\centering}\gnumboxLO{\ninerm{brclr 6}}
	&\tableOmPB{\centering}\gnumboxLO{\ninerm{bclr 6}}
	&\tableOmPB{\centering}\gnumboxLB{\ninerm{bms}}
	&\tableOmPB{\centering}\gnumboxLO{\ninerm{tst}}
	&\tableOmPB{\centering}\gnumbox{\ninerm{tstA}}
	&\tableOmPB{\centering}\gnumbox{\ninerm{tstX}}
	&\tableOmPB{\centering}\gnumboxLP{\ninerm{tst}}
	&\tableOmPB{\centering}\gnumboxSB{\ninerm{tst}}
	&\tableOmPB{\centering}\gnumboxLY{\ninerm{ts}}
	&\tableOmPB{\centering}\gnumbox{\ninerm{-}}
	&\tableOmPB{\centering}\gnumbox{\ninerm{nop}}
	&\tableOmPB{\centering}\gnumboxLB{\ninerm{bsr}}
	&\tableOmPB{\centering}\gnumboxLO{\ninerm{jsr}}
	&\tableOmPB{\centering}\gnumbox{\ninerm{jsr}}
	&\tableOmPB{\centering}\gnumbox{\ninerm{jsr}}
	&\tableOmPB{\centering}\gnumbox{\ninerm{-}}
	&\tableOmPB{\centering}\gnumboxLP{\ninerm{jsr}}
	&\tableOmPB{\centering}\gnumbox{\ninerm{-}}
	&\tableOmPB{\centering}\gnumboxLY{\ninerm{jsr}}
\\
	 \multicolumn{1}{|p{\tableOmColB}|}%
	{\tableOmPB{\centering}\gnumbox{\ninerm{{\color{ghost}x}E}}}
	&\tableOmPB{\centering}\gnumboxLO{\ninerm{brset 7}}
	&\tableOmPB{\centering}\gnumboxLO{\ninerm{bset 7}}
	&\tableOmPB{\centering}\gnumboxLB{\ninerm{bil}}
	&\tableOmPB{\centering}\gnumbox{\ninerm{-}}
	&\tableOmPB{\centering}\gnumboxY{\ninerm{mov}}
	&\tableOmPB{\centering}\gnumboxO{\ninerm{movp}}
	&\tableOmPB{\centering}\gnumboxP{\ninermb{mov}}
	&\tableOmPB{\centering}\gnumbox{\ninerm{-}}
	&\tableOmPB{\centering}\gnumboxDG{\ninermb{xmov}}
	&\tableOmPB{\centering}\gnumbox{\ninerm{stop}}
	&\tableOmPB{\centering}\gnumbox{\ninermb{*}}
	&\tableOmPB{\centering}\gnumboxLG{\ninerm{ldX}}
	&\tableOmPB{\centering}\gnumboxLO{\ninerm{ldX}}
	&\tableOmPB{\centering}\gnumbox{\ninerm{ldX}}
	&\tableOmPB{\centering}\gnumbox{\ninerm{ldX}}
	&\tableOmPB{\centering}\gnumbox{\ninerm{ldX}}
	&\tableOmPB{\centering}\gnumboxLP{\ninerm{ldX}}
	&\tableOmPB{\centering}\gnumboxSB{\ninerm{ldX}}
	&\tableOmPB{\centering}\gnumboxLY{\ninerm{ldX}}
\\
	 \multicolumn{1}{|p{\tableOmColB}|}%
	{\tableOmPB{\centering}\gnumbox{\ninerm{{\color{ghost}x}F}}}
	&\tableOmPB{\centering}\gnumboxLO{\ninerm{brclr 7}}
	&\tableOmPB{\centering}\gnumboxLO{\ninerm{bclr 7}}
	&\tableOmPB{\centering}\gnumboxLB{\ninerm{bih}}
	&\tableOmPB{\centering}\gnumboxLO{\ninerm{clr}}
	&\tableOmPB{\centering}\gnumbox{\ninerm{clrA}}
	&\tableOmPB{\centering}\gnumbox{\ninerm{clrX}}
	&\tableOmPB{\centering}\gnumboxLP{\ninerm{clr}}
	&\tableOmPB{\centering}\gnumboxSB{\ninerm{clr}}
	&\tableOmPB{\centering}\gnumboxLY{\ninerm{clr}}
	&\tableOmPB{\centering}\gnumbox{\ninerm{wait}}
	&\tableOmPB{\centering}\gnumbox{\ninerm{tXA}}
	&\tableOmPB{\centering}\gnumboxLG{\ninerm{aiX}}
	&\tableOmPB{\centering}\gnumboxLO{\ninerm{stX}}
	&\tableOmPB{\centering}\gnumbox{\ninerm{stX}}
	&\tableOmPB{\centering}\gnumbox{\ninerm{stX}}
	&\tableOmPB{\centering}\gnumbox{\ninerm{stX}}
	&\tableOmPB{\centering}\gnumboxLP{\ninerm{stX}}
	&\tableOmPB{\centering}\gnumboxSB{\ninerm{stX}}
	&\tableOmPB{\centering}\gnumboxLY{\ninerm{stX}}
\\
\hhline{|-|~~~~~~~~~~~~~~~~~~~}
\caption{\label{OpcodeSummary68hc908}68hc908 opcode summary}
\end{longtable}
%%}


%% \tableOmEnd


%
}%
\end{landscape}%
\restoregeometry     %so it does not affect the rest of the pages.

\ifthenelse{\isundefined{\languageshorthands}}{}{\languageshorthands{\languagename}}

